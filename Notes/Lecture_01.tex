% Created 2020-08-25 ter 12:02
% Intended LaTeX compiler: pdflatex
\documentclass[11pt]{article}
\usepackage[utf8]{inputenc}
\usepackage{lmodern}
\usepackage[T1]{fontenc}
\usepackage[top=2cm, bottom=2cm, left=2cm, right=2cm]{geometry}
\usepackage{graphicx}
\usepackage{longtable}
\usepackage{float}
\usepackage{wrapfig}
\usepackage{rotating}
\usepackage[normalem]{ulem}
\usepackage{amsmath}
\usepackage{textcomp}
\usepackage{marvosym}
\usepackage{wasysym}
\usepackage{amssymb}
\usepackage{amsmath}
\usepackage[theorems, skins]{tcolorbox}
\usepackage[style=abnt,noslsn,extrayear,uniquename=init,giveninits,justify,sccite,
scbib,repeattitles,doi=false,isbn=false,url=false,maxcitenames=2,
natbib=true,backend=biber]{biblatex}
\usepackage{url}
\usepackage[linktocpage,pdfstartview=FitH,colorlinks,
linkcolor=blue,anchorcolor=blue,
citecolor=blue,filecolor=blue,menucolor=blue,urlcolor=blue]{hyperref}
\usepackage{attachfile}
\usepackage{setspace}
\usepackage{tikz}
\usepackage{minted}
\author{Gabriel Petrini}
\date{August 25th, 2020}
\title{Course Intro}
\begin{document}

\maketitle
\tableofcontents


\section{Introduction}
\label{sec:orge0f02e7}

\subsection{What this class is about}
\label{sec:org695ceb8}

This class is a smattering of advanced econometrics topics.

\begin{itemize}
\item Understanding the usefulness of structural modeling
\item Learning the computational tools for estimating structural models
\item Advanced topics in treatment effects and measurement error models
\end{itemize}


\subsection{Applicability of topics}
\label{sec:orgf24f91e}

The techniques we will cover are used in a wide variety of fields of applied microeconomics:

\begin{itemize}
\item Labor
\item Education
\item IO
\item Public
\item Development
\item Health
\item Urban/Regional
\item Environmental
\item Others
\end{itemize}


\subsection{What we will cover in the class}
\label{sec:orgec16462}

\begin{enumerate}
\item Basic computing and things you need to think about
\item Coding, version control, reproducibility, workflow
\item Estimating and simulating structural models
\item Subjective expectations models
\item Measurement error correction
\item Treatment effects
\item Machine learning
\end{enumerate}


\subsection{Grading}
\label{sec:org7791d9b}

\begin{itemize}
\item Problem sets:
\begin{itemize}
\item You must use Julia and write .jl scripts, no Jupyter
\item You can work in groups of up to 3, but you must turn in your own code
\end{itemize}
\item Class participation:
\item Midterm exam:
\item Paper presentation:
\begin{itemize}
\item You must consult with me at least 1 week prior to your scheduled presentation date to ensure the paper is appropriate for a presentation
\end{itemize}
\item Paper referee report:
\begin{itemize}
\item The paper shouldn't be published, or if it has been published, you should use the earliest pre-print version
\end{itemize}
\item Research proposal:
\end{itemize}

\section{More about Julia}
\label{sec:org9afcdd4}

\subsection{Basic operations}
\label{sec:orgfc6c4da}

\begin{itemize}
\item \textbf{Array indexing:} use \texttt{[ ]}
\item \textbf{Show output:} use \texttt{println()}
\item \textbf{Commenting:} use \texttt{\#} for single line, \texttt{\#= ... =\#} for multi-line
\item \textbf{Element-wise operators:} must put a \texttt{.} in front, e.g. \texttt{x .+ y} if \texttt{x} and \texttt{y} are arrays
\item \textbf{Load installed package:} \texttt{using Random}
\item \textbf{Execute script:} \texttt{include("myfile.jl")}
\end{itemize}

\section{Next lecture}
\label{sec:org70b29e4}
\end{document}
