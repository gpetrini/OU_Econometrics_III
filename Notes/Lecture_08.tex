% Created 2020-09-29 ter 18:34
% Intended LaTeX compiler: pdflatex
\documentclass[11pt]{article}
\usepackage[utf8]{inputenc}
\usepackage{lmodern}
\usepackage[T1]{fontenc}
\usepackage[top=3cm, bottom=2cm, left=3cm, right=2cm]{geometry}
\usepackage{graphicx}
\usepackage{longtable}
\usepackage{float}
\usepackage{wrapfig}
\usepackage{rotating}
\usepackage[normalem]{ulem}
\usepackage{amsmath}
\usepackage{textcomp}
\usepackage{marvosym}
\usepackage{wasysym}
\usepackage{amssymb}
\usepackage{amsmath}
\usepackage[theorems, skins]{tcolorbox}
\usepackage[style=abnt,noslsn,extrayear,uniquename=init,giveninits,justify,sccite,
scbib,repeattitles,doi=false,isbn=false,url=false,maxcitenames=2,
natbib=true,backend=biber]{biblatex}
\usepackage{url}
\usepackage[cache=false]{minted}
\usepackage[linktocpage,pdfstartview=FitH,colorlinks,
linkcolor=blue,anchorcolor=blue,
citecolor=blue,filecolor=blue,menucolor=blue,urlcolor=blue]{hyperref}
\usepackage{attachfile}
\usepackage{setspace}
\usepackage{tikz}
\author{Gabriel Petrini}
\date{October 1st, 2020}
\title{Estimating Dynamic Models Without Solving Value Functions}
\begin{document}

\maketitle
\tableofcontents


\bibliography{References}


\section*{\cite{hotzMiller1993}}
\label{sec:org9c92db5}

Dynamic discrete choice models are complicated to estimate because of the future value terms. \citet{hotzMiller1993} show:

\begin{itemize}
\item Differences in conditional value functions \(v_j-v_{j'}\) can be mapped into \uline{conditional choice probabilities} ( \(p_j\)'s )
\item We can pull the \(p_j\)'s from the data in a first stage
\item \textbf{Empirical example:} optimal stopping with respect to couples' fertility
\end{itemize}

\subsection*{Difference in \(v\)'s and logit errors}
\label{sec:orgc62eac5}

Consider an individual who faces two choices where the errors are T1EV. The probability of choice 1 is:
\begin{align*}
p_1&=\frac{\exp(v_1)}{\exp(v_0)+\exp(v_1)}
\end{align*}

The ratio of \(p_1/p_0\) is then:
\begin{align*}
\frac{p_1}{p_0}&=\frac{\exp(v_1)}{\exp(v_0)} = \exp(v_1 - v_0)
\end{align*}
implying that:
\begin{align*}
\ln(p_1/p_0)&=v_1-v_0
\end{align*}

\subsection*{General structure}
\label{sec:org916f1e4}

The inversion theorem of Hotz and Miller says that there exists a mapping, \(\psi\), from the conditional choice probabilities, the \(p\)'s, into the differences in the conditional valuation functions, \(v_j-v_k\):
\begin{align*}
V_{t+1}&=v_{0t+1}+\mathbb{E}\max\{\epsilon_{0t+1},v_{1t+1}+\epsilon_{1t+1}-v_{0t+1},...,\\
&\phantom{\text{----}}v_{{J}t+1}+\epsilon_{{J}t+1}-v_{0t+1}\}\\
V_{t+1}&=v_{0t+1}+\mathbb{E}\max\{\epsilon_{0t+1},\psi_0^1(p_{t+1})+\epsilon_{1t+1},...,\psi_0^{{J}}(p_{t+1})+\epsilon_{{J}t+1}\}
\end{align*}

The \(p\)'s can be taken from the data.  However:

\begin{enumerate}
\item We need the mapping, \(\psi\),
\item We need to be able to calculate the expectations of the \(\epsilon\)'s
\item We need to do something with the \(v_0\)'s
\end{enumerate}
\end{document}